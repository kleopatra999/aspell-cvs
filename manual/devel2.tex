%% LyX 1.1 created this file.  For more info, see http://www.lyx.org/.
%% Do not edit unless you really know what you are doing.
\documentclass[11pt,english]{article}
\usepackage[T1]{fontenc}
\usepackage[latin1]{inputenc}
\usepackage{babel}
\setlength\parskip{\medskipamount}
\setlength\parindent{0pt}

\makeatletter

%%%%%%%%%%%%%%%%%%%%%%%%%%%%%% LyX specific LaTeX commands.
\providecommand{\LyX}{L\kern-.1667em\lower.25em\hbox{Y}\kern-.125emX\@}

%%%%%%%%%%%%%%%%%%%%%%%%%%%%%% Textclass specific LaTeX commands.
 \newenvironment{lyxcode}
   {\begin{list}{}{
     \setlength{\rightmargin}{\leftmargin}
     \raggedright
     \setlength{\itemsep}{0pt}
     \setlength{\parsep}{0pt}
     \normalfont\ttfamily}%
    \item[]}
   {\end{list}}

%%%%%%%%%%%%%%%%%%%%%%%%%%%%%% User specified LaTeX commands.

\usepackage[T1]{fontenc}
\usepackage{xspace}
\newcommand{\nach}{$\to$\xspace}
\newcommand{\hoch}{\texttt{$^\wedge$}}

\usepackage{html}

\newcommand{\doubledash}{-\hspace{0.1em}-}
\newcommand{\doubledashb}{-\/-}
\newcommand{\dlt}{{\footnotesize$\ll$}}
\newcommand{\dgt}{{\footnotesize$\gg$}}

\begin{htmlonly}

\renewenvironment{lyxcode}
  {\begin{list}{}{
    \setlength{\rightmargin}{\leftmargin}
    \raggedright
    \setlength{\itemsep}{0pt}
    \setlength{\parsep}{0pt}
    \ttfamily}%
   \item[] 
   \begin{ttfamily}}
   {\end{ttfamily}
    \end{list} }

\newenvironment{LyXParagraphIndent}[1]%
{\begin{quote}}
{\end{quote}}

\renewcommand{\LyX}{LyX}

\renewcommand{\doubledash}{\rawhtml &#45;&#45;\endrawhtml}
\renewcommand{\doubledashb}{\rawhtml &#45;&#45;\endrawhtml}
\renewcommand{\dlt}{�}
\renewcommand{\dgt}{�}

\renewcommand{\nach}{\rawhtml <i>to</i> \endrawhtml}
\renewcommand{\hoch}{\rawhtml &#94;\endrawhtml}

\end{htmlonly}

\makeatother
\begin{document}

\title{Aspell Devel Docs}


\author{Kevin Atkinson\\
kevin at atkinson dhs org}

\maketitle
\tableofcontents{}


\section*{Notes}

This manual is designed for those who which to developer Aspell. It
is currently very sketchy. However, it should improve over time.

The eventual goal is to convert this manual into Texinfo. However,
since I do not have the time to learn Texinfo right now, I decided
to use something I am already conferable with. Once someone goes through
the trouble of converting it into Texinfo I will maintain the Texinfo
version.


\section{Style Guidelines}

As far as coding styles go I am really not that picky. The important
thing is to stay consistent. However, please what ever you do, do
not indent with more than 4 characters as I find indenting with more
than that extremely difficult to read as most of the code ends up
on the right side of the window.


\section{C++ Standard Library}

The C++ Standard library is not used directly except under very specific
circumstances. The string class and the STL is used indirectly though
wrapper classes and all I/O is done using the standard C library with
light right helper classes to make using C I/O a bit more C++ like.

However the new, new{[}{]}, delete and delete{[}{]} operates are used
to allocated memory when appropriate.


\section{Templates}

Templates are used in Aspell when there is a clear advantage to doing
so. When ever you use templates please use them carefully and try
very hard not to create code bloat by generating a lot of unnecessary,
and duplicate code.


\section{Error Handling}

Exceptions are not used in Aspell as I find them more trouble than
they are worth. Instead an alternate method of error handling is used
which is based around the PosibErr class. PosibErr is a special Error
handling device that will make sure that an error is properly handled.
It is defined in {}``posib\_err.hpp''. PosibErr is expected to be
used as the return type of the function It will automatically convert
to the \char`\"{}normal\char`\"{} return type however if the normal
returned type is accessed and there is an \char`\"{}unhandled\char`\"{}
error condition it will abort It will also abort if the object is
destroyed with an \char`\"{}unhandled\char`\"{} error condition. This
includes ignoring the return type of a function returning an error
condition. An error condition is handled by simply checking for the
presence of an error, calling ignore, or taking ownership of the error.

The PosibErr class is used extensively though out Aspell. Please refer
to the Aspell source for examples of using PosibErr until better documentation
is written.


\section{Source Code Layout }

\begin{description}
\item [common/]Common code used by all parts of Aspell
\item [lib/]Library code used only by the actual Aspell library
\item [data/]Data files used by Aspell
\item [modules/]Aspell modules which are eventually meant to be pluggable

\begin{description}
\item [speller/]~

\begin{description}
\item [default/]Main speller Module. The speller-util program is also included
in this directory
\end{description}
\item [filter/]~
\item [tokenizer/]~
\end{description}
\item [auto/]Scripts and data files to automatically generate code used
by Aspell
\item [interface/]Header files and such that external programs should use
when in order to use the Aspell library.

\begin{description}
\item [cc/]The external {}``C'' interface that programs should be using
when they wish to use Aspell.
\end{description}
\item [prog/]Actual programs based on the Aspell library. The main {}``aspell''
utility is included here.
\item [scripts/]Misc. scripts used by Aspell
\item [manual/]~
\item [examples/]Example programs deminstating the use of the Aspell library
\end{description}

\section{Strings}


\subsection{String}

The String class provided the same functionally of the C++ string
except for fewer constructors. It also inherits OStream so that you
can write to it with the {}``<\/<'' operator. It is defined in {}``string.hpp''.


\subsection{ParmString}

ParmString is a special string class that is designed to be used as
a parameter for a function that is expecting a string. It is defined
in {}``parm\_sting.hpp''. It will allow either a \char`\"{}const
char {*}\char`\"{} or \char`\"{}String\char`\"{} class to be passed
in. It will automatically convert to a \char`\"{}const char {*}\char`\"{}.
The string can also be accesses via the \char`\"{}str\char`\"{} method.
Usage example:

\begin{lyxcode}
void~foo(ParmString~s1,~ParmString~s2)~\{~\\
~~~const~char~{*}~str0~=~s1;~\\
~~~unsigned~int~size0~=~s2.size()~\\
~~~if~(s1~==~s2~||~s2~==~\char`\"{}bar\char`\"{})~\{~\\
~~~~~...~\\
~~~\}~\\
\}~\\
...~\\
String~s1~=~\char`\"{}...\char`\"{};~\\
foo(s1);~\\
const~char~{*}~s2~=~\char`\"{}...\char`\"{};~\\
foo(s2);
\end{lyxcode}
This class should be used when a string is being passed in as a parameter.
It is faster than using {}``const String \&'' (as that will create
an unnecessary temporary when a const char {*} is passed in), and
is less annoying than using {}``const char {*}'' (as it doesn't
require the c\_str() method to be used when a String is passed in).


\subsection{CharVector}

A character vector is basically a Vector<char> but it has a few additional
methods for dealing with strings which Vector does not provide. It,
like String, is also inherits OStream so that you can write to it
with the {}``<\/<'' operator. It is defined in {}``char\_vector.hpp''.
Use it when ever you need a string which is guaranteed to be in a
continuous block of memory which you can write to.


\section{Smart Pointers}

Smart pointers are used extensively in Aspell to avoid simply memory
management takes and to avoid memory leaks.


\subsection{CopyPtr}

The CopyPtr class makes a deep copy of an object when ever it is copied.
The CopyPtr class is defined in {}``copy\_ptr.hpp''. This header
should be included where ever CopyPtr is used. The complete definition
of the object CopyPtr is pointing to does not need to be defined at
this point. The implementation is defined in {}``copy\_ptr-t.hpp''.
The implementation header file should be included at a point in your
code where the class CopyPtr is pointing to is completely defined.


\subsection{ClonePtr}

ClonePtr is like copy pointer except the clone() method is used instead
of the copy constructor to make copies of an object. If is defined
in {}``clone\_ptr.hpp'' and implemented in {}``clone\_ptr-t.hpp''.


\subsection{StackPtr}

A StackPtr is designed to be used when ever the only pointer to a
new object allocated with \textbf{new} is on the stack. It is similar
to the standard C++ auto\_ptr but the semantics are a bit different.
It is defined in {}``stack\_ptr.hpp'' unlike CopyPtr of ClonePtr
it is defined and implemented in this header file.


\subsection{GenericCopyPtr}

A generalised version of CopyPtr and ClonePtr which the two are based
on. It is defined in {}``generic\_copy\_ptr.hpp'' and implemented
in {}``generic\_copy\_ptr-t.hpp''.


\section{I/O}

Aspell does not use C++ I/O classes and function in any way since
they do not provide a way to get at the underlying file number and
can often be slower than the highly tuned C I/O functions found in
the standard C library. However, some light weight wrapper classes
are provided to that standard C I/O can be used in a more C++ like
way.


\subsection{IStream/OStream}

These two base classes mimic some of the functionally of the C++ functionally
of the corresponding classes. They are defined in {}``istream.hpp''
and {}``ostream.hpp'' respectfully. They are however based on standard
C I/O and are not proper C++ streams.


\subsection{FStream}

Defined in {}``fstream.hpp''


\subsection{Standard Streams}

CIN/COUT/CERR. Defined in {}``iostream.hpp''.


\section{Config Class}

The Config class is used to hold configuration information. It has
a set of keys which it will except. Inserting or even trying to look
at a key that it does not know will produce an error. It is defined
in {}``common/config.hpp''


\section{Filter Interface}

The current filter interface in Aspell is very simple and somewhat
limited. Basically a filter is given a block of text and its job is
to blank out parts that should not be spell checked. Filers are not
allowed to do anything else but blank text out. Currently all filters
must be compiled in but my eventual goals are to allow filters to
be loaded at run time.

To add a new filter create a new file in the modules/filter directory,
the file should be a C++ file and end in {}``.cpp''. The file should
contain a new filter class inherited from IndividualFilter, a function
to return a new filter, and an optional KeyInfo array for adding options
to control the behaviour of the filter. The file then needs to be
added to Makefile.am so that the build system knows about the filter
and lib/new\_filter.cpp must be modified so that Aspell knows about
the filter.


\subsection{IndividualFilter class}

All filters are required to inherit from the IndividualFilter class
found in {}``indiv\_filter.hpp''. See that file for more details
and the other filter modules for example of how it is used.


\subsection{Constructor Function}

After the class is created a function must to created which will return
a new filter allocated with \textbf{new}. The function must have the
following prototype:

\begin{lyxcode}
IndividualFilter~{*}~new\_<\,{}<filter\_name>\,{}>
\end{lyxcode}

\subsection{Config Options}

A filter may have any number of options associated with it as long
as they all start with the filter name. See the \TeX{} and SGML filter
for examples of what to do and {}``config.hpp'' for the definition
of the KeyInfo struct.


\subsection{Makefile Modifications}

After the new file is created simply add the file to the {}``libaspell\_filter\_standard\_la\_SOURCES''
line in {}``modules/filter/Makefile.am'' so that the build system
knows about it.


\subsection{New\_filter Modifications}

Finally modify {}``lib/new\_filter.cpp'' so that Aspell knows about
the new filter. Follow the example there for the other filter modules.
The filter\_modules array should only be modified if there your filter
has config options.


\section{Data Structures}

When ever possible you should try to use on of the data structures
available. If the data structures do not provide enough functionally
for your needs you should consider enhancing them rather than written
something from scratch.


\subsection{Vector}

The vector class is defined in {}``vector.hpp'' and works the same
way as the standard STL vector does except that it doesn't have as
many constructors.


\subsection{BasicList}

BasicList is a simple list structure which can either be implemented
as a singly or doubly linked list. It is defined in {}``basic\_list.hpp''.


\subsection{StringMap}

StringMap is a associative array for strings. You should try to use
this when ever possible to avoid code bloat. It is defined in {}``string\_map.hpp''


\subsection{Hash Tables}

Several hash tables are provided when StringMap is not appropriate.
These hash tables provide a hash\_set, hash\_multiset, hash\_map and
hash\_multimap which are very similar to SGI STL's implementation
with a few exceptions. It is defined in {}``hash.hpp''


\subsection{BlockSList}

BlockSList provided a pool of nodes which can be used for singly linked
lists. It is defined in {}``block\_slist.hpp''.


\section{Mk-Src Script}

A good deal of interface code is automatically generated by the {}``mk-src.pl''
Perl script. I am doing it this way to avoid having to write a lot
of relative code for the C++ interface. This should also make adding
interface for other languages a lot less tedious and will allow the
interface to automatically take advantage of new Aspell functionality
as it is made available. The {}``mk-src.pl'' script uses {}``mk-src.in''
as its input.
\end{document}
